\section{Simulaciones}
\label{sec:simulaciones}

\subsection{Dimensiones}
\label{subsec:acdim}
Las simulaciones 2D se enfocarán en analizar el comportamiento de células dentro de una grilla 2D de $100 \times 100$ celdas.

Para las simulaciones 3D se trabajará dentro de una grilla 2D de $50 \times 50 \times 50$ celdas.

La grilla 2D contiene una región interna (núcleo) de $10 \times 10$ celdas, en la cual se generan las condiciones iniciales, mientras que la 3D tiene un núcleo de $20 \times 20 \times 20$.
Dentro de esta región, se toman distintos porcentajes de células vivas iniciales en el intervalo $(0, 100)$ y para cada uno de estos porcentajes, se realizan 10 simulaciones.

\subsection{Condiciones iniciales}
\label{subsec:acini}
Para cada porcentaje elegido se generaron 10 condiciones iniciales distintas, generando dicha cantidad de celdas vivas aleatoriamente en dentro del núcleo.

\subsubsection{Condición de corte}
\label{subsubsec:ac2corte}
El máximo número de iteraciones es de 100.
La simulación se detiene antes si alguna célula activa alcanza el borde de la grilla.

\subsection{Masa}
Para cada porcentaje, en cada una de las reglas, se realizó un análisis de la evolución de la cantidad de células vivas en función de la iteración, denominada masa.
En cada gráfico se puede ver el cambio de la masa para las 10 simulaciones.

\subsection{Radio}
De manera análoga a la masa, se realizaron gráficos analizando el cambio del radio, que se definió como la mayor distancia (norma 1) entre una célula y el centro de la grilla.

\subsection{Observables}
Debido a que las condiciones iniciales se tomaron de manera aleatoria y los resultados finales dependian de las mismas, los gráficos muestran los valores promedio del observable en cada porcentaje (para las 10 condiciones iniciales de cada uno).

El promedio se calculó de la siguiente manera:

\begin{equation}
    \label{eq:mean}
    \bar{m} = \frac{1}{10} \sum_{i=1}^{10} m_i
\end{equation}
donde $m_i$ es la masa de la simulación $i$

El desvio estandar se obtuvo de la siguiente manera:

\begin{equation}
    \label{eq:std}
    \sigma = \sqrt{\frac{1}{9} \sum_{i=1}^{10} (m_i - \bar{m})^2}
\end{equation}
donde $\bar{m}$ es el promedio previamente definido

\subsubsection{Masa final}
Para todas las reglas tanto 2D como 3D, se eligió tomar como observable la masa final del sistema, es decir la que se alcanza en la iteración numero 100, o cuando se toca un borde.
Así, se graficó el valor del observable en función de las iteraciones.
\subsubsection{Generación de estabilización}
Además, para las reglas donde se observó que la masa en un ciclo estable, se eligió tomar como otro observable a la generación en la que alcanza dicho ciclo.