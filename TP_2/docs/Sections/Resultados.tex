\newcommand{\massgraph}[1]{
    \begin{figure}[H]
        \centering
        \includegraphics[width=0.65\textwidth]{Images/#1/#1_mass}
        \caption{Gráfico de la masa en función del tiempo}
        \label{fig:#1mass}
    \end{figure}
}

\newcommand{\radiusgraph}[1]{
    \begin{figure}[H]
        \centering
        \includegraphics[width=0.65\textwidth]{Images/#1/#1_radius}
        \caption{Gráfico del radio en función del tiempo}
        \label{fig:#1radius}
    \end{figure}
}

\newcommand{\obsfinalgraph}[1]{
    \begin{figure}[H]
        \centering
        \includegraphics[width=0.65\textwidth]{Images/#1/#1_obs_final}
        \caption{Gráfico de la masa final en función de los porcentajes}
        \label{fig:#1obsfinal}
    \end{figure}
}

\newcommand{\obsstablegraph}[1]{
    \begin{figure}[H]
        \centering
        \includegraphics[width=0.65\textwidth]{Images/#1/#1_obs_stable}
        \caption{Gráfico de la generación de estabilización en función de los porcentajes}
        \label{fig:#1obsstable}
    \end{figure}
}

\section{Resultados}
\label{sec:resultados}

\subsection{Autómatas celulares de dos dimensiones}
\label{subsec:results2d}
\subsubsection{El juego de la vida de Conway (Von Neumann)}
\massgraph{gol2Dv2}
Se puede observar en la Fig. \ref{fig:gol2Dv2mass} que la masa tiene una tendencia decreciente y se estabiliza en un valor constante, existiendo algunos casos donde dicho valor no es $0$.
\radiusgraph{gol2Dv2}
Se puede observar en la Fig. \ref{fig:gol2Dv2radius} que el radio exhibe tres comportamientos: mantenerse constante, decrecer hasta anularse o hasta un valor constante no nulo.
\obsfinalgraph{gol2Dv2}
Se puede observar en la Fig. \ref{fig:gol2Dv2obsfinal} que la masa final de cada simulación tiene una tendencia creciente en función de los porcentajes. Parece alcanzar estabilidad al rededor del 80\%.
Sin embargo, el desvío asociado con estos porcentajes también es muy grande.
\obsstablegraph{gol2Dv2}
Se puede observar en la Fig. \ref{fig:gol2Dv2obsstable} que la generación de estabilización tiene una tendencia creciente en función de los porcentajes hasta el 80\%, y luego disminuye.
Sin embargo, estos últimos valores también tienen un desvío muy grande.


\subsubsection{Vecinos impares (Moore)}
\massgraph{odd2D}
Se puede observar en la Fig. \ref{fig:odd2Dmass} que si bien el cambio de la masa es algo caótico, parece haber una tendencia global creciente.
Además se puede ver que hay cierta concordancia en las generaciones donde ocurren caídas en la masa. La más destacable es la que ocurren al rededor de la generación 30, donde todos los porcentajes exhibidos sufren una caída de la masa.

\radiusgraph{odd2D}
A pesar de lo irregular que es el crecimiento de la masa, en la Fig. \ref{fig:odd2Dradius} se ve que los radios en función de la generación crecen de manera constante y sin saltos.
\obsfinalgraph{odd2D}
Por último, en la Fig. \ref{fig:odd2Dobsfinal} se ve que la masa final de cada simulación en función del porcentaje tiene una tendencia creciente que va desacelerando.

\subsubsection{Rectángulo Relleno}
\massgraph{fill2D}
Se puede observar en la Fig. \ref{fig:fill2Dmass} que la masa es creciente, lo cual se puede explicar con la regla que establece que si la celda ya estaba viva antes, entonces sigue viva en la siguiente iteración.
Y luego de algunas iteraciones se estabiliza en un valor constante.
\radiusgraph{fill2D}
Al igual que la masa, se puede observar en la Fig. \ref{fig:fill2Dradius} que el radio también es creciente, y luego de algunas iteraciones se estabiliza en un valor constante.
\obsfinalgraph{fill2D}
En la Fig. \ref{fig:fill2Dobsfinal} se puede observar que la masa de cada simulación tiene una tendencia creciente en los primeros porcentajes, y a partir de 35\% se queda constante en 100, que coincide con la superficie del cuadrado del núcleo.
\obsstablegraph{fill2D}
En la Fig. \ref{fig:fill2Dobsstable} se puede observar que la generación de estabilización crece hasta el 20\%, y luego decrece.


\subsection{Autómatas celulares de tres dimensiones}
\label{subsec:results3d}

\subsubsection{El juego de la vida de Conway 3D (Moore)}
\massgraph{gol3D}
En la Fig. \ref{fig:fill2Dmass} se puede ver una tendencia global creciente de la masa. Además se puede ver que los porcentajes mayores que 12.5\% sufren una caída inicial y luego entran en la tendencia creciente.
También cabe destacar que todos los porcentajes terminan por la condición de corte en vez de llegar a la iteración máxima establecida.

\radiusgraph{gol3D}
Se puede observar en la Fig. \ref{fig:fill2Dradius} que el radio tiene una tendencia global creciente, y a diferencia de la masa, las caídas son menos frecuentes y de menor magnitud.

\obsfinalgraph{gol3D}
Se puede observar que el valor final aumenta hasta alcanzar lo que parecería un máximo local en 37.5\%, y luego disminuye.
Es interesante notar como en los primeros y en los últimos valores, el desvío estandar aumenta, mientras que acercandose al máximo disminuye.


\subsubsection{Umbral (Moore)}
\massgraph{decay3D}
Se puede observar en la Fig. \ref{fig:decay3Dmass} que para los primeros porcentajes (1.25\% y 12.5\%), la masa tiende a decrecer hasta llegar a $0$.
Los que ocupan entre el 25\% y el 57.5\% del núcleo, tienen una tendencia creciente.
Por último, el que ocupa el 93.75\% del núcleo sufre una caída inicial, y luego crece y alcanza una masa cíclica.
\radiusgraph{decay3D}
En la Fig. \ref{fig:decay3Dradius}, se puede observar que el radio tiene un comportamiento similar a la masa.
Las simulaciones con 1.25\% y 12.5\% tienen una tendencia decreciente hasta llegar a $0$.
Aquellas simulaciones que cuentan con un porcentaje entre el 25\% y el 75\% del núcleo tienen una tendencia creciente.
Por último, el que ocupa el 93.75\% del núcleo sufre una caída inicial, y luego crece y alcanza un radio constante.
\obsfinalgraph{decay3D}
En la Fig. \ref{fig:decay3Dobsfinal} se puede observar que la masa final es 0 para los porcentajes 1.25\% y 12.5\%, y entre 25\% y 62.5\% se mantiene cercano a los 20000, y luego comienza a decrecer. Vale destacar que para 93.75\% el desvío es bastante más bajo que el resto.

\subsubsection{Umbral (Von Neumann)}
\massgraph{decay3Dv2}
En la Fig. \ref{fig:decay3Dv2mass}
\radiusgraph{decay3Dv2}
\obsfinalgraph{decay3Dv2}
\obsstablegraph{decay3Dv2}