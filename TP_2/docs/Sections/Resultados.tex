\newcommand{\massgraph}[1]{
    \begin{figure}[H]
        \centering
        \includegraphics[width=0.65\textwidth]{Images/#1/#1_mass}
        \caption{Gráfico de la masa en función del tiempo}
        \label{fig:#1mass}
    \end{figure}
}

\newcommand{\radiusgraph}[1]{
    \begin{figure}[H]
        \centering
        \includegraphics[width=0.65\textwidth]{Images/#1/#1_radius}
        \caption{Gráfico del radio en función del tiempo}
        \label{fig:#1radius}
    \end{figure}
}

\newcommand{\obsfinalgraph}[1]{
    \begin{figure}[H]
        \centering
        \includegraphics[width=0.65\textwidth]{Images/#1/#1_obs_final}
        \caption{Gráfico de la masa final en función de los porcentajes}
        \label{fig:#1obsfinal}
    \end{figure}
}

\newcommand{\obsstablegraph}[1]{
    \begin{figure}[H]
        \centering
        \includegraphics[width=0.65\textwidth]{Images/#1/#1_obs_stable}
        \caption{Gráfico de la generación de estabilización en función de los porcentajes}
        \label{fig:#1obsstable}
    \end{figure}
}

\section{Resultados}
\label{sec:resultados}

\subsection{Autómatas celulares de dos dimensiones}
\label{subsec:results2d}
\subsubsection{El juego de la vida de Conway (Von Neumann)}
\massgraph{gol2Dv2}
Se puede observar en la Fig. \ref{fig:gol2Dv2mass} que la masa tiene una tendencia decreciente y se estabiliza en un valor constante, existiendo algunos casos donde dicho valor no es $0$.
\radiusgraph{gol2Dv2}
En la Fig. \ref{fig:gol2Dv2radius}, el radio exhibe tres comportamientos: mantenerse constante, decrecer hasta anularse o hasta un valor constante no nulo.
\obsfinalgraph{gol2Dv2}
Se puede observar en la Fig. \ref{fig:gol2Dv2obsfinal} que la masa final de cada simulación tiene una tendencia creciente en función de los porcentajes. Parece alcanzar estabilidad al rededor del 80\%.
Sin embargo, el desvío asociado con estos porcentajes también es muy grande.
\obsstablegraph{gol2Dv2}
Se puede observar en la Fig. \ref{fig:gol2Dv2obsstable} que la generación de estabilización tiene una tendencia creciente en función de los porcentajes hasta el 80\%, y luego disminuye.
Sin embargo, estos últimos valores también tienen un desvío muy grande.


\subsubsection{Vecinos impares (Moore)}
\massgraph{odd2D}
En la Fig. \ref{fig:odd2Dmass} se ve que si bien el cambio de la masa es bastante irregular, parece haber una tendencia global creciente.
Además se puede ver que hay cierta concordancia en las generaciones donde ocurren caídas en la masa, siendo la más destacable la que ocurre alrededor de la generación 30, donde todos los porcentajes exhibidos sufren una rápida caída de la masa, para luego volver a subir.

\radiusgraph{odd2D}
A pesar de lo irregular que es el crecimiento de la masa, en la Fig. \ref{fig:odd2Dradius} se ve que los radios en función de la generación crecen de manera constante y sin saltos.
\obsfinalgraph{odd2D}
Por último, en la Fig. \ref{fig:odd2Dobsfinal} se ve que la masa final de cada simulación en función del porcentaje tiene una tendencia creciente que va desacelerando.

\subsubsection{Rectángulo Relleno}
\massgraph{fill2D}
Se puede observar en la Fig. \ref{fig:fill2Dmass} que la masa es creciente, lo cual se puede explicar con la regla que establece que si la celda ya estaba viva antes, entonces sigue viva en la siguiente iteración.
Y luego de algunas iteraciones se estabiliza en un valor constante.
\radiusgraph{fill2D}
Al igual que la masa, se puede observar en la Fig. \ref{fig:fill2Dradius} que el radio también es creciente, y luego de algunas iteraciones se estabiliza en un valor constante.
\obsfinalgraph{fill2D}
En la Fig. \ref{fig:fill2Dobsfinal} se ve que la masa de cada simulación tiene una tendencia creciente en los primeros porcentajes (con un desvío estandar muy grande entre el 10\% y el 20\%), y a partir de 35\% alcanza un valor constante de 100,
que coincide con la masa del núcleo en la Ec. \ref{eq:fill2Dmasscore}.
\begin{equation}
    \label{eq:fill2Dmasscore}
    m_{nucleo2D} = {L_{nucleo2D}}^2 = 10^2 = 100
\end{equation}
\obsstablegraph{fill2D}
La Fig. \ref{fig:fill2Dobsstable} parece indicar que la generación de estabilización crece hasta el 20\%, y luego decrece (aunque el desvio estandar es bastante grande en gran parte del crecimiento y decrecimiento).


\subsection{Autómatas celulares de tres dimensiones}
\label{subsec:results3d}

\subsubsection{El juego de la vida de Conway 3D (Moore)}
\massgraph{gol3D}
En la Fig. \ref{fig:gol3Dmass} se puede ver una tendencia global creciente de la masa.
Por otro lado, los porcentajes mayores que 12.5\% sufren una caída inicial y luego entran en la tendencia creciente.
También cabe destacar que todos los porcentajes terminan por la condición de corte en vez de llegar a la iteración máxima establecida.

\radiusgraph{gol3D}
Se puede observar en la Fig. \ref{fig:gol3Dradius} que el radio tiene una tendencia global creciente, y a diferencia de la masa, las caídas son menos frecuentes y de menor magnitud.

\obsfinalgraph{gol3D}
En la Fig. \ref{fig:gol3Dobsfinal} se puede observar que el valor final aumenta hasta alcanzar lo que parecería un máximo local en 37.5\%, y luego disminuye.
Es interesante notar como en los primeros y en los últimos valores, el desvío estandar aumenta, mientras que acercandose al máximo disminuye.


\subsubsection{Umbral (Moore)}
\massgraph{decay3D}
Se puede observar en la Fig. \ref{fig:decay3Dmass} que para los primeros porcentajes (1.25\% y 12.5\%), la masa tiende a decrecer hasta llegar a $0$.
Los que ocupan entre el 25\% y el 87.5\% del núcleo, tienen una tendencia creciente.
Por último, el que ocupa el 93.75\% del núcleo sufre una caída inicial, y luego crece y alcanza una masa cíclica.
\radiusgraph{decay3D}
En la Fig. \ref{fig:decay3Dradius}, se puede observar que el radio tiene un comportamiento similar a la masa.
Las simulaciones con 1.25\% y 12.5\% tienen una tendencia decreciente hasta llegar a $0$.
Aquellas simulaciones que cuentan con un porcentaje entre el 25\% y el 87.5\% del núcleo tienen una tendencia creciente.
Por último, el que ocupa el 93.75\% del núcleo sufre una caída inicial, y luego crece y alcanza un radio constante.
\obsfinalgraph{decay3D}
En la Fig. \ref{fig:decay3Dobsfinal} se puede observar que la masa final es 0 para los porcentajes 1.25\% y 12.5\%, y entre 25\% y 62.5\% se mantiene cercano a los 20000, y luego comienza a decrecer. Vale destacar que para 93.75\% el desvío es significativamente más bajo que el resto.

\subsubsection{Umbral (Von Neumann)}
\massgraph{decay3Dv2}
En la Fig. \ref{fig:decay3Dv2mass} se puede observar que salvo el caso de 1.25\%, que decrece hasta llegar a un estado estable cercano a 0, el resto de los porcentajes alcanza un estado estable alrededor de 4000, que coincide con la mitad de la masa del núcleo expresada en la Ec. \ref{eq:decay3Dv2masscore}.
\begin{equation}
    \label{eq:decay3Dv2masscore}
    \frac{m_{nucleo3D}}{2} = \frac{{L_{nucleo3D}}^3}{2} = \frac{20^3}{2} = 4000
\end{equation}
También vale destacar que los porcentajes mayores o iguales a 50\% sufren una caída inicial, y luego crecen hasta llegar al equilibrio, mientras que los menores a 50\% (sin incluir al 1.25\%) no sufren dicha caída.


\radiusgraph{decay3Dv2}
En la Fig. \ref{fig:decay3Dv2radius}, de manera similar a la masa, se puede observar que para el 1.25\%, el radio disminuye hasta estabilizarse por debajo de 25, mientras que el resto de los porcentajes se estabilizan en 30, que coincide con el radio máximo que puede tener un punto dentro del cubo del núcleo (una esquina), expresado en la Ec. \ref{eq:decay3Dv2maxdist}.
\begin{equation}
    \label{eq:decay3Dv2maxdist}
    r_{esquina} = 3 \cdot \frac{L_{nucleo3D}}{2} = 3 \cdot \frac{20}{2} = 30
\end{equation}
\obsfinalgraph{decay3Dv2}
En la Fig. \ref{fig:decay3Dv2obsfinal} se confirma lo que se observó en la Fig. \ref{fig:decay3Dv2mass}, donde las masas finales de todos los porcentajes excepto 1.25\% se estabilizan cerca de 4000, mientras que para 1.25\% se estabiliza cerca de 0. Cabe destacar que el desvío en general es bajo.
\obsstablegraph{decay3Dv2}
En la Fig. \ref{fig:decay3Dv2obsstable} se puede observar que para las simulaciones con 1.25\% la generación se estabiliza muy rápidamente, mientras que para el resto, la generación de estabilización tiene un mínimo en 50\% (cuya masa es 4000, la masa aproximada donde se estabiliza) y parece aumentar cada vez más a medida que se aleja del 50\%. Cabe destacar también, que los porcentajes que son complementarios entre sí (suman 100\%) tienen una generación de estabilización cercana.

\subsection{Conclusiones}
\label{subsec:conclusiones}
El análisis de los autómatas celulares de dos y tres dimensiones permitió observar el comportamiento emergente resultante de simples reglas.
Se pudo ver que dicho comportamiento es difícil de predecir.

Si bien algunos autómatas celulares exhibieron ser altamente dependientes de las condiciones iniciales, mostrando que pequeñas variaciones en las mismas pueden llevar a comportamientos muy distintos, al mismo tiempo, se observó que para ciertas reglas, las condiciones iniciales no afectan tanto el comportamiento final del sistema, como es el caso de la regla de Rectángulo Relleno o Umbral (Von Neumann).

Es interesante ver como, a pesar de que el comportamiento de los autómatas celulares es tan diverso y aparentemente caótico, es posible definir observables que permiten caracterizar las distintas condiciones iniciales.
