\section{Implementación}
\label{sec:implementacion}

El simulador se desarrolló en el lenguaje de programación Java, versión 18.
Las clases principales del proyecto son \texttt{Main}, \texttt{FileController}, \texttt{CelularAutomata2D}, \texttt{CelularAutomata3D} y \texttt{AutomatonRules}.

\subsection{Clase \texttt{Main}}
\label{subsec:main}
En \texttt{Main} se definen la cantidad de celdas vivas iniciales para cada simulación, la cantidad de generaciones a simular y el tipo de autómata celular a utilizar.
Además, se inicializa el autómata celular y se ejecuta la simulación.

Trabaja con matrices de booleanos para representar el estado de las celdas en cada generación.

Estas se leen de archivos de texto de nombre \texttt{init\{2|3\}d\_[cantidad de celdas iniciales].txt} ubicados en la carpeta \texttt{input}.
El resultado de la simulación se guarda en distintos archivos de texto en la carpeta \texttt{output}, dentro de subcarpetas con el nombre del tipo de autómata celular.

\subsection{Clase \texttt{FileController}}
\label{subsec:filecontroller}
\texttt{FileController} se encarga de leer y escribir carpetas y archivos de texto.

\subsubsection{Lectura}
\label{subsubsec:lectura}
Es responsable de obtener los estados iniciales de los autómatas, a partir de los archivos de texto mencionados en la sección anterior.
El formato de los archivos es el siguiente:
\begin{itemize}
    \item La primera línea contiene un entero que representa el lado la grilla cuadrada.
    \item La segunda linea contiene un entero que indica el lado del cuadrado interno que contiene las celdas vivas iniciales.
    \item Las siguientes líneas contienen cada una, una matriz de booleanos, donde \texttt{0} representa una celda muerta y \texttt{1} una celda viva.
    \item En total hay 10 matrices, para tener 10 variantes de una misma cantidad de celulas iniciales vivas.
\end{itemize}

\subsubsection{Escritura}
\label{subsubsec:escritura}
Es responsable de escribir cada uno de los estados de las celdas en cada generación en archivos de texto.
El formato de los archivos es el siguiente:
\begin{itemize}
    \item La primera línea contiene un entero que representa el lado la grilla cuadrada.
    \item La segunda linea contiene un entero que indica el lado del cuadrado interno que contiene las celdas vivas iniciales.
    \item La tercera linea se deja en blanco.
    \item Las siguientes líneas contienen cada una, una matriz de booleanos, donde \texttt{0} representa una celda muerta y \texttt{1} una celda viva.
    \item Cada linea representa una generación de la simulación.
\end{itemize}

\subsection{Clases \texttt{CelularAutomata2D} y \texttt{CelularAutomata3D}}
\label{subsec:celularautomata}
Estas clases se encargan de simular el autómata celular en 2 y 3 dimensiones, respectivamente.

Ofrecen métodos para evolucionar el sistema en una generación y para obtener la cantidad de vecinos vivos de una celda en particular, usando distancia de Moore y Von Neumann.

También permiten saber si una celda está viva o muerta en una generación dada, si una posición se encuentra dentro de los límites de la grilla y si se alcanzó la condición de corte de contacto con el borde.

\subsection{Clase \texttt{AutomatonRules}}
\label{subsec:automatonrules}
\texttt{AutomatonRules} define las reglas de los autómatas celulares implementados.
Se implementa a partir de dos enums, \texttt{Rules2D} y \texttt{Rules3D}, que contienen las reglas de los autómatas celulares de dos y tres dimensiones, respectivamente.

Cada regla tiene asociada una función que determina si una celda debe estar viva o muerta en la siguiente generación, además de un nombre corto para identificar la regla.